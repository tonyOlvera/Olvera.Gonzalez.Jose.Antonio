\documentclass[10pt,a4paper]{report}
\usepackage[latin1]{inputenc}
\usepackage[spanish]{babel}
\usepackage{amsmath}
\usepackage{amsfonts}
\usepackage{amssymb}
\usepackage{graphicx}
\usepackage[left=2cm,right=2cm,top=2cm,bottom=2cm]{geometry}
\author{Jose Antonio Olvera Gonzalez }
\title{ Par de rotacion y cuaternios}

\begin{document}
\begin{center}
 Par de rotacion
\end{center}
\begin{flushleft}
Son  de rotacion con respecto a el eje que se decea depende de matrices Rx, Ry y Rz representa cada uno de los ejes en diferentes direcciones, en una matris de rotacion lo que se consigue es que partiendo de un sistema de cordenadas definido (x, y, z) es un sistema hoctonormal ya que todos los ejes tienen un angulo de 90 grados, eso es lo que define a un espacio en 3 dimenciones.
\end{flushleft}
\begin{center}
Cuaternios
\begin{flushleft}
Los cuaterniones(tambien llamados cuaternios) son una extencion de los numeros reales, similar a la de los numeros complejos. Mientras que los numeros complejos son una extencion de los reales por la adicion de la unidad imaginaria,los cuarteniones sin la extencion generada de manera analogica.
\subsection{Representacion de los cuarteniones }
\begin{flushleft}
\textbf{Vectorial:}
\begin{flushleft}
Son numeros reales univocamente determinados por cada cuaternion. Analogicamente pueden expresarse como el producto interno( componente a componencte) de dos vectores.
\begin{flushleft}
\textbf{Matricial:}
\begin{flushleft}
Hay almenos dos formas, isomorfismos para representar cuarteniones como matrices usando matrises complejas de 2x2, los cuartetos unitarios juegan un papel importante. Una propiedad interesante de esta representacion es que todos los numeros complejos son matrices que solo tiene compolnentes reales  usando matrices de 4x4.
\subsection{Aritmetica basica de cuarteniones}
\textbf{Parte rel e imaginaria de un cuartenion}

\begin{flushleft}
Se convierte en un numero real si todas las otras coordenadas es igual a cero. De modo que tal eje real esta contenido en el conjunto de todos los cuarteniones. 
\begin{flushleft}
\textbf{Adicion}

\begin{flushleft}
La adicion se realiza analogicamente a como se hace con los complejos, es decir, termino a termino.
\begin{flushleft}
\textbf{Producto}
\begin{flushleft}
El producto se realiza componente a componente, y esta dado en su forma completa.
\begin{flushleft}
\textbf{Conjugacion}
\begin{flushleft}
El conjugado interviene el signo de los componentes ( agregados) del cuartenion. Matricialmente esto correspondera a la operacion de trasposicion de cualquiera de sus reresentaciones matriciales.
\begin{flushleft}
La medida o valor absolutode un cuaternion.
\begin{flushleft}
\textbf{Cociente}
\end{flushleft}

\begin{flushleft}
El inverso multiplicativo de un cuaternion distinto de cero el cual es el mismo patron que cumplen los numeros ccomplejos.
\begin{flushleft}
\textbf{Exponenciacion}

\begin{flushleft}
La exponenciacion de numeros cuaternionicos, al igual que sucede con los numeros complejos, esta relacionada con funciones trigonometricas, Dado un cuaternion escrito en forma canonica.
\begin{flushleft}
\textbf{Comparacion con matrices}
\begin{flushleft}
La multiplicacion de matrices no es en general conmutativa al igual que en el caso de los cuaterniones. Sin embargo, tampoco todas las matrices poseen un universo multipliativo mientras que todos los cuaternios diferentes del cero son intertibles 
\end{flushleft}
\end{flushleft}
\end{flushleft}
\end{flushleft}
\end{flushleft}
\end{flushleft}
\end{flushleft}
\end{flushleft}
\end{flushleft}
\end{flushleft}
\end{flushleft}
\end{flushleft}
\end{flushleft}

\end{flushleft}
\end{flushleft}
\end{flushleft}
\end{flushleft}
\end{flushleft}
\end{center}
\end{document}