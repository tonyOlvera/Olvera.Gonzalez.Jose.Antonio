\documentclass[12pt,letterpaper]{report}
\usepackage[utf8]{inputenc}
\usepackage[spanish]{babel}
\usepackage{amsmath}
\usepackage{amsfonts}
\usepackage{amssymb}
\usepackage{graphicx}
\usepackage[left=2cm,right=2cm,top=2cm,bottom=2cm]{geometry}
\author{Jose Antonio Olvera Gonzalez }
\title{Primer avance}
\begin{document}
\begin{center}
\includegraphics[scale=1]{logo.jpg}\\

Nombres:\\ Cruz Camacho Diego
		Olvera Gonzalez Jose Antonio\\
		Arreola Vazquez Jesus Alberto\\
		Asencio Neri Fernando\\
		Medina Negrete Joshua Isaac\\
Carrera: Ing.Mecatronica\\
Grado y Grupo: 7ºB\\
Materia:Cinematica de Robot\\
Profesor: Moran garabito Carlos Enrique
		
		
		
\chapter{INDICE}
\end{center}	




\section{Identificacion de necesidad}
\section{Investigacion de antecedentes}
\section{Enunciado de objetivo}
\section{Espeficaciones de la tarea}
\section{Sintesis}
\section{Analisis}
\section{Selecciòn}
\section{Diseñado y detallado}
\section{Prototipos y pruebas}
\section{Producciòn} 


\begin{flushleft}
\begin{flushleft}
\part{}

\textbf{1.1 Identificacion de necesidades}
\end{flushleft}
\begin{flushleft}
Explicare mi proyecto explicando el area donde se trabaje y el como se hara. El proyecto sera especialmente para el area de medicina, algun tratamiento o terapia, es el conjunto de medios cuya finalidad es la reabilitacion de enfermedades o alguna paralisis en el cuerpo a travez de distintas formas. Es un tipo de juicio clinico, son sinonimos de terapia, terapèutico, cura y metodo curativo.
\begin{flushleft}
No se debe confundir con terapèutica, que es la rama de las ciencias de la salud que se ocupa de los medios empleados y su forma de aplicarlos en el tratamiento de las enfermedades, con el fin de aliviar los sintomas o de producirl curacion.
\begin{flushleft}
Pero existen ciertos campos donde no puede se una reavilitacion completa ya que puede se dificil o por el hecho de ir en realbilitacion o el simple tralsado puede complicar la reavilitacion ya que no se tiene un dezcanso total, por eso creamos una cubierta o manga de exoesqueleto donde facìlite el movimiento del brazo en recuperacion.\\
Usarla sera una ayuda terpeuthica a base de una ayuda de la electronica y robotica.
\begin{flushleft}
El cual nos permite brindar apoyo atravez de  reabilitacion con mecanismos roboticos basados en el movimiento de brazo con una velocidad ajustable, y a los movimientos no son complicados de hacer.
\begin{flushleft}
\textbf{1.2 Investigacion de antecedentes}
\begin{flushleft}
De acuerdo con el reporte del Banco mundial sobre discapacidad presentado en 2015, alrededor de 15 porciento de la poblacion mundial (cerca de 1000millones de personas) experimentan alguna forma de discapacidad. siendo los paìses en desarrollo los mas afectados por esta problemàtica. Segùn el censo 2010 del Instituto Nacional de Estadisticas y Geografìa(INEGI), en Mèxico mas del 5 porciento de la poblacion presenta algun tipo de discapacidad, siendo los problemas de movilidad los que representan mas del 58 porciento de las discapacidades. El aumento del nùmero de las personas que padecen alguna discapacidad motora o debilidad muscular a motivado el desarrollo de la robotica de rehabilitacion. El objetivo de esta rama emergente de la robotica de servicio es la aplicacion de la tecnologia para el desarrollo de dispositivos que asistan y mejoren las terapias de rehabilitacion para las personas con discapacidad.
\begin{flushleft}
En las terapias fìsicas convencionales se establecen actividades que son allevadas a cabo por el fisioterapeuta. Èsta depende del grado de discapacidad del paciente y de la parte del cuerpo afectada. El objetivo de la rehabilitacion fìsica es rehabilitar al paciente de alguna deficiencia o discapacidad para mejorar su movilidad de la parte afectada para tener una mejor calidad de vida. Sin embargo, se presentan añgunos problemas en la reabilitacion y estàn presentes tanto para el pasiente como para el terapeuta. Para el paciente las terapias son costosas y el tiempo de recuperacion es largo, en el caso del especialista se presenta problemas de eficiencia( debido al desgaste fisico), a demàs no cuenta con medidas fiables del deterioro del miembro a rehabilitar, ni tampoco con medidas para evaluar el progreso de la rehabilitaciòn implementada.
\begin{flushleft}
La rehabilitación asistida con robots tiene el potencial de superar algunas de las limitaciones de los métodos convencionales y puede favorecer el desarrollo de nuevos tipos de tratamientos de rehabilitación. La terapia asistida con robots puede proporcionar una rehabilitación intensiva de larga duración, sin ser afectada por las habilidades y el nivel de fatiga del terapeuta. Además, puede reducir los costos de la terapia a largo plazo y proporcionar datos cuantitativos para evaluar el progreso de los pacientes.
\begin{flushleft}
Los exoesqueletos son sistemas robóticos que se acoplan al cuerpo humano de forma externa para cumplir funciones específicas y forman parte de un grupo denominado Wearable Robots (robots usables), estos sistemas son usados por una persona, de tal manera que la interfaz física conduce a una transferencia directa de energía mecánica y al intercambio de información. Los exoesqueletos están diseñados para coincidir con la forma y la función del cuerpo humano .En años recientes, los exoesqueletos se han empleado como dispositivos orientados a la rehabilitación física. Se presentan amplias revisiones del estado del arte de exoesqueletos desarrollados para la rehabilitación del miembro superior. Algunos exoesqueletos para rehabilitación de miembro superior ya se encuentran disponibles comercialmente, por ejemplo: Aupa, JACE S600, JACE S603, Armeo® Spring, Armeo® Spring Pediatric, Armeo® Boom, Armeo® Power. Las principales desventajas de estos dispositivos son: las medidas antropométricas en las cuales se basa su diseño, no corresponden a las de la población mexicana y su alto costo de adquisición y de mantenimiento los hacen inaccesibles para la mayoría de las instituciones de salud en México, principalmente para aquellas que prestan sus servicios en zonas marginadas.
\begin{flushleft}
\textbf{1.3 Enunciado de objetivo}
\begin{flushleft}
El objetivo del proyecto se basa en brindar apoyo usando conocimientos de electronica y robotica a la rama terapetica sobre apoyo de movimientos que puedan ser complicados.
\begin{flushleft}
\textbf{1.4 Especificaciones de la tarea}
\begin{flushleft}

1.- Conocer la problemática de los movimientos.\\

2.- Hacer calcúlos y parametros para poder adaptar el movimiento.\\

3.- Buscar una forma de hacer que el guante pueda ser comodo para el usuario.\\

4.- Empezar con el prototipado.
\begin{flushleft}
\textbf{1.5 Sintesis}
\begin{flushleft}
De acuerdo a lo antes mencionado, se pretende hacer un mecanismo donde la aplicación y conocimiento de robotica puede influir en la reabilitación de un braso humano.
\end{flushleft}
\begin{flushleft}
Por tanto sedara comienzo al proyecto anual que con la asesoria de Ingenieros de la UPZMG se podra realizar lo señalado en un lapso de un año.
\begin{flushleft}
Cabe mencionar que se añadirán las siguientes materias al proyecto de la manga relabilitadora para englobar las materias en un solo proyecto, las cuales son: Termodinámica, Diseño y selección de elementos mecánicos, Modelado y simulación de sistemas y Cinemática de robots.
\begin{flushleft}
\textbf{1.6 Analisis}
\begin{flushleft}
1.	Medio ambiente de el proyecto:
El proyecto se relaizara en la UPZMG con las mejoras correspondientes a cada cuatrimestre.\\

2.	Renatbilidad:
El proyecto sera rentable, ya que investigando antecedentes y ahora en la actualidad no se encuentra algun aparato de las mismas caracteristicas en el mercado.
En la comunidad Universutatia de la UPZMG es muy rentable para los y las compañeras de la carrera de terapia fisica.\\

3.	Necesidades de mercado:
El mercado necesita aparatos sofisticados, que puedan cubrir la demanda. El principal inconveniente es el precio del producto el cual seria un tanto elevado por los materiales a utilizar, como por ejemplo los servomotores, sensores, y la programación.\\

4.	Factibilidad politica:
Es factible en el aspecto que es legal, no se plagearia ya que no existe otro igual.\\

5.	Aceptación cultural:
Muchas personas no estan familiarizadas con la nueva tecnologia por consecuencia se dara una explicación de como funciona el mecanismo, que tan seguro puede ser,  que personas lo pueden usar y desde  que edades.\\

6.	Medio físico:
El mecanismo lo podemos utilizar en un entorno urbano o rural, como por ejemplo en la ciudad como en un pueblo.\begin{flushleft}

\end{flushleft}
\begin{flushleft}
\textbf{1.7 Seleccion}
\begin{flushleft}
Selección de materiales:\\
Los materiales a utilizar son: Plastico, barras de alumnio, cableado de cobre, poleas de acero, raspberry.\\
Selección de softwares:\\
Los software a utilizar son:
Matlab(Para la realización de calculos matematicos)
Autocad(Para la realización de Piezas a utilizar)
Orcad (Para la realizacion de circuitos electrónico)
Latex(Para la reportes de avances)
Gazebo(Para el entorno grafico y simulaciones de robot)
Ros(Para programacion de robo)
Blender(Para el modelado, iluminacion, renderizado, animacion y creacion de graficos) 

\end{flushleft}
\end{flushleft}



\end{flushleft}
\end{flushleft}



\end{flushleft}
\end{flushleft}
\end{flushleft}
\end{flushleft}
\end{flushleft}
\end{flushleft}
\end{flushleft}
\end{flushleft}
\end{flushleft}
\end{flushleft}
\end{flushleft}
\end{flushleft}


\end{flushleft}
\end{flushleft}
\end{flushleft}
\end{flushleft}
\end{flushleft}


		
\end{document}